% Clase del texto con el tamaño de la fuente.
\documentclass[11pt, twoside, letterpaper]{IEEEtran}

% Tamaño del papel y de las margenes
\usepackage{geometry}
\geometry{
    letterpaper,
    top=0.75in, 
    bottom=1in, 
    left=0.68in, 
    right=0.68in,
    marginparwidth=0.5in,
}

% Se selecciona el idioma y se redefine el termino "Cuadro" por Tabla y "Resumen" por Abstract.
\usepackage[spanish]{babel}
\renewcommand{\spanishtablename}{Tabla}
\renewcommand{\spanishabstractname}{Abstract}
\selectlanguage{spanish}

% Codificación a utilizar.
\usepackage[utf8]{inputenc}
\usepackage[T1]{fontenc}
\usepackage{lmodern}

% Se corrige silabas de algunas palabras.
\hyphenation{pa-la-bra pa-rra-fo}

% Para modificar el Titulo
\usepackage{titling}
\pretitle{} \posttitle{}
\preauthor{} \postauthor{}
\predate{} \postdate{}

% Se selecciona el directorio donde se encuentran las imágenes.
\usepackage{graphicx, wrapfig, float}
\graphicspath{{images/}}

% Para ajustar el interlineado.
\usepackage{setspace}

% Para ajustar la sangría de la primera linea.
\setlength{\parindent}{3em}

% Para ajustar el espacio entre párrafos.
\setlength{\parskip}{2ex}

% Paquetes matemáticos.
\usepackage{amssymb, amsmath, amsbsy, mathdots, mathrsfs, cancel}

% Para listas personalizadas.
\usepackage{enumitem}

% Para las letras capitales.
\usepackage{lettrine}

% Encabezado y pie de página

% Configuración de los encabezados y los pie de página.
\usepackage{fancyhdr}
\pagestyle{fancy}
\fancyhf{}
% Elementos en la cabecera [par]{impar}
\lhead[\thepage]{}
\chead[]{}
\rhead[]{\thepage}
\renewcommand{\headrulewidth}{0.0pt}

% Elementos en el pie de página [par]{impar}
\lfoot[]{}
\cfoot[]{}
\rfoot[]{}
% \renewcommand{\footrulewidth}{0.5pt}


% Primera página
\fancypagestyle{plain}{
    \fancyhead[L]{}
    \fancyhead[C]{}
    \fancyhead[R]{}
    \fancyfoot[L]{}
    \fancyfoot[C]{}
    \fancyfoot[R]{}
    \renewcommand{\headrulewidth}{0.0pt}
    %\renewcommand{\footrulewidth}{0.5pt}
}

% Textos de prueba.
\usepackage{blindtext}

% Enlaces.
\usepackage{hyperref}

% Paquete gráfico
\usepackage{tikz}

% Referencias.
\usepackage[maxbibnames=99, sorting=none, backend=biber, style=ieee]{biblatex}
\addbibresource{references.bib}